\section{Introduction}

\tinyos is an open-source operating system designed to 
support the networked sensors, a platform regime with
severe hardware restrictions especially with respect to memory and power. 
The \tinyos runtime environment provides not only the traditional
operating system role of resource mediation and system libraries, but
also encompasses a set of support tools for data collection and
aggregation, and debugging assistance. 

\tinyos originally began as a research project at UC Berkeley in 1999
with the goal of providing a platform for embedded systems. Beginning
in 1999 Berkeley developed the first of a series of wireless sensor
nodes, commonly called motes. \tinyos was the operating system
created by then graduate student Jason Hill for the Berkeley-designed
motes \cite{jhill-thesis}. While \tinyos is still a UC Berkeley research project, today
there are now hundreds of companies and institutions using \tinyos as
their embedded systems platform, the most common of which is networked
embedded sensors or \emph{motes}. There are several instantiations of
\tinyos radio stacks (Bluetooth, 802.15.4, proprietary FSK) as well as several
processor ports (Atmel Atmega128, Motorola HCS08, TI MSP340).  The
sensors that the motes offer vary from mote to mote but can include
sensors for magnetic field, acceleration, light, sound, temperature,
barometric pressure and relative humidity.  The motes \tinyos
currently supports include the Mica and Mica2 (Atmega128/Chipcon
CC1000) and the Telos (TI MSP340/Chipcon CC2420) mote families.

Due to the variety among platform sensor and communication capabilities,
this document only covers the \tinyos components common to all platforms.
Platform-specific guides are available to cover platform-specific
components.\footnote{No they're not.}

{\bf About this document.} This document assumes a basic understanding
of programming tools including shell scripts, cvs, compilers, the C
programming language, and make. If you do not understand one of 
these tools, try using \emph{Google} to
find more information. We've also listed some references at the back 
of the document.

\subsection{nesC}
\tinyos is written in a C extension Language called \nesc that provides
language support to the concepts and execution model of \tinyos . The
\nesc compiler, \ncc, preprocesses \tinyos code before passing
compilation on to the Gnu C Compiler, gcc. \nesc allows for more
complete error reporting, java-like interface syntax, programming
model support. 

Though this manual does not explain in detail \nesc syntax or semantics, 
\nesc will be explained where necessary for understanding. For maximum
benefit, we highly recommend that the \nesc Reference Manual be 
read before continuing beyond this introduction.

{\bf Programming conventions.} 
enum. 














