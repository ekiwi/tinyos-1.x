>
\section{Communication}

\subsection{Communication Model: Active Messages}

Active message model. Use IDs to dispatch messages to message handler
code. Refer to AM paper.

Active message model used regardless of the underlying communication
substrate: radio, UART (serial).

\subsubsection{Buffer model}

\emph{Sending.}
To avoid potentially expensive memory management schemes in the AM
layer but still allow concurrent operation, TinyOS Active Message
buffers follow a strict alternating ownership protocol. Provided that the 
messaging layer returns SUCCESS from the send() call, the active
message layer 'owns' the send buffer provided by the higher-level
component until
the send is complete as indicated by the sendDone() event. That is, 
the requesting component should not modify the buffer a between the
successful send() call and the corresponding sendDone() event. 

The send-buffer management simplicity usually requires that the
application implement some buffer management. A common, simple mechanism is a
pending flag which keeps track of a single buffer's status. If the previous
message is still being sent, the application cannot modify the buffer
and must return FAIL for any send requests.  If the flag indicates
that the send buffer is available, the application is free to fill the
buffer and send the message. Of course more complicated schemes can be
implemented where necessary.

\emph{Receiving.} AM memory management for incoming messages requires
that the receiving component return a pointer to a free message
buffer; the buffer can be the same buffer recieved in that call or a
different buffer. A message arrives and fills a buffer, and then the Active
Message layer dispatches the message based on the handler type. The buffer
is handed to the application component (through the
ReceiveMsg.receive() event), and  the application
component must return a pointer to a free buffer upon completion.

If the application is done with the buffer, it can return a pointer to
that buffer. If your component needs to save the message contents for
later use, it needs to copy the message to a new buffer, or return a
new (free) message buffer for use by the network stack.

\subsection{Packet definition}

TOS_Msg defined in AM.h, but some platforms redefine TOS_Msg to make,
for example, address handling or byte-order conversion more easy. In
this case the definition is in tos/platforms/<platform>/AM.h, but the
definitions remain very similar.

Put definition in a nice box.

typedef struct TOS_Msg
{
  /* The following fields are transmitted/received on the radio. */
  uint16_t addr;
  uint8_t type;
  uint8_t group;
  uint8_t length;
  int8_t data[TOSH_DATA_LENGTH];
  uint16_t crc;

  /* The following fields are not actually transmitted or received 
   * on the radio! They are used for internal accounting only.
   * The reason they are in this structure is that the AM interface
   * requires them to be part of the TOS_Msg that is passed to
   * send/receive operations.
   */
  uint16_t strength;
  uint8_t ack;
  uint16_t time;
  uint8_t sendSecurityMode;
  uint8_t receiveSecurityMode;  
} TOS_Msg;

Slightly different fields for TinySec packets. See TinySec section.

TOSMsg fields (table). 
Columns would be field/semantics/when set and by whom
addr 
type

TOSH_DATA_LENGTH. TOSH_DATA_LENGTH is a constant that defines the 
\emph{maximum} length of the data payload in the messages. When 
message buffers are allocated, the are generally done so by declaring
a TOS_Msg structure. Rather than introduce potentially complicated 
code that can handle data packets of arbitrary length, the TOS_Msg
structure allocates space for the data using the TOSH_DATA_LENGTH 
constant. 

The actual length of
the data payload is in the length field of the TOSMsg structure. The 
communications components can handle transmission/receipt of the 
actual data using the length field.

There are hooks in the TinyOS make system (tools/make) that allow
you can to set the value of TOS_DATA_LENGTH in your application by 
defining MSG_SIZE to the size you'd like in your application's makefile
where MSG_SIZE is the value you want to use for TOSH_DATA_LENGH.
For example, 'MSG_SIZE=40' would effectively change TOSH_DATA_LENGTH to 
40 bytes. 


\subsection{Communication Interfaces}
\subsubsection{SendMsg}
\subsubsection{ReceiveMsg}

\subsubsection{Mac-layer Control, Reliability & Error Detection}

Always call radio control (SetRFPower(), MacControl.enableAck(), etc.)
functions in StdControl.start() rather than StdControl.init().  The
order of init() calls in TinyOS cannot be predetermined, so the radio's init()
function may be called after the application's init() fucntion and
reset any previously set settings.

CRCs.

interface MacBackoff
{
  async event int16_t initialBackoff(TOS_MsgPtr m);
  async event int16_t congestionBackoff(TOS_MsgPtr m);
}

Acks (CC1000 & CC2420 below, RFM?)

interface MacControl
{
  async command void enableAck();
  async command void disableAck();
}

Acks are disabled by default. MacControl.enableAck() must be called
before message is sent. If on, hardware will automatically send an
acknowledgement.  If acknowledgements are enabled, the radio stack
will wait for an ack or timeout before signaling the sendDone event.
If an ack is received, the TOSMsg.ack field will be set to 1, it will
be set to 0 otherwise.

Note: cc1000 requires 250us to switch radio's mode from rcv to xmit,
then 10-15 bytes to sync sending radio with receiver radio, and only
then can you send the ACK. The effective throughput cut by
1/3. Nevertheless, ACKs can be sent on CC1000 (Is this still true?)

\subsection{Achievable Bandwidth}

\subsection{Communication Components}

\subsection{GenericComm}

SendMsg, ReceiveMsg, StdControl. 

\subsubsection{Low-level}
\subsubsection{RadioCoordinator}
\subsubsection{BareSendMsg}

\subsection{Communication Implementations}
\subsubsection{UART}

\subsubsection{CC1000Radio}

CC1000RadioControl

\subsubsubsection{PowerManagement}

App needs to call radio component's StdControl.stop() function. For
  example: CC2420RadioC.StdControl.stop()

\subsubsection{CC2420 Radio}

High-level overview
Design goals.

High-level use by applications:
CC2420RadioC 
  - BareSendMsg
  - ReceiveMsg
  - CC2420Control
  - MacControl
  - MacBackoff

(wired  to CC2420RadioM, CC2420RadioControlM)

Mention that the sendDone and receive signals are signaled from
the stack rather than from the radio interrupt contexts. Semantics
could be important (i.e., task that signals sendDone behind other 
long-running tasks).

\subsubsubsection{CC2420 Implementation}

A note about the mapping between spec, CC2420Const.h, and usage in
HPL* files. Current parameters stored in a current parameter (CP)
array.

{\bf Initialization.} Done in split-phase. 
Question: How deprecated is stdcontrol? That is, are we actively 
porting all the StdControl() interfaces to SplitControl()? Or is this
isolated to radio stack?

SplitControl used because some of the init/start/stop calls 
require delays. For example, 

SplitControl and CC2420Control interfaces implemented by CC2420ControlM.

{\bf Buffering.}: two receive buffers, one send buffer. 
Question: In receive, the implementation checks for a valid 
returned buffer (that is, a non-null pointer) and only sets
bPacketReceiving to FALSE if it's non-null. If it is null, doesn't
that completely screw up receiving? That is, isn't there not another
way to give a receive buffer back to the stack?

State machines: 
timer
radio
    DISABLED_STATE = 0, (stopped, as in app called CC2420RadioM.StdControl.stop())
    IDLE_STATE, 
    TX_STATE, (attempting to transmitting (possibly must wait for radio
      to switch to xmit or for SFD pin to go high)
    TX_WAIT, 
    PRE_TX_STATE, (if we're backed-off in a congestion timer)
    POST_TX_STATE,
    POST_TX_ACK_STATE,
    RX_STATE,
    POWER_DOWN_STATE,
    WARMUP_STATE, (just called start)

    TIMER_INITIAL = 0,
    TIMER_BACKOFF,
    TIMER_ACK

Send: if the radio is in a state where it can transmit, wait for SFD
(Start of Frame Delimeter) to go high. When the pin goes high, set
state to TX_STATE . Will retry up to 8 times; reasons for not 
being able to send include: 

sfd.captured: 
- could be one of several states:
- 

Send failed situation: if a send fails, sendDone is signaled w/ FAIL,
pointer to txbuffer (duh).

uses {
    interface SplitControl as CC2420SplitControl;
    interface CC2420Control;
    interface HPLCC2420 as HPLChipcon;
    interface HPLCC2420FIFO as HPLChipconFIFO; 
    interface HPLCC2420Interrupt as FIFOP;
    interface HPLCC2420Capture as SFD;
    interface StdControl as TimerControl;
    interface TimerJiffyAsync as BackoffTimerJiffy;
    interface Random;
    interface Leds;
  }
}

\subsubsection{Hardware Resources}
 
Timer1.

Timer2.

\subsubsubsection{PowerManagement}





