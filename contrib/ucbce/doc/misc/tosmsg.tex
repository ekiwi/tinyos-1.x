
\documentclass[10pt]{article}


%%%%%%%%%%%%%%%%%%%%%%%%%%%%%%%%%%%%

\setlength{\topmargin}{0.in}
\setlength{\textheight}{8.5in}  % Use for proofreading.
\setlength{\textwidth}{6.5in}
\setlength{\oddsidemargin}{0.in}


\renewcommand{\textfraction}{.1}
\renewcommand{\bottomfraction}{.3}
\setlength{\textfloatsep}{13pt}
\setlength{\abovedisplayskip}{8pt}
\setlength{\belowdisplayskip}{8pt}



%\usepackage{endfloat}
%\usepackage{doublespace}
%%%  Use for initial submission, remove for final.
\usepackage{fancyhdr}


\lhead{D. M. Doolin}
\chead{}
\rhead{Sensor power}
\lfoot{\today}
\cfoot{}
\rfoot{\thepage}
\renewcommand{\headrulewidth}{0.4pt}
\renewcommand{\footrulewidth}{0.4pt}
\pagestyle{fancy}


\input {comment}
\usepackage{chicago}
\usepackage{draftcopy}
\usepackage{subfigure}
\usepackage{psfig}
\usepackage{graphicx}
\usepackage{amssymb}
\usepackage{amstext,amsmath}
\usepackage{latexsym}
\usepackage{mathrsfs}  %  command: \mathscr{#}
\usepackage{psfrag}
\usepackage{comment}
\usepackage{url}
\usepackage{algorithm}
\usepackage{algorithmic}
\usepackage{lscape}
\usepackage{color}


\newcommand{\eq}{Eq.}
\newcommand{\eqs}{Eqs.}
\newcommand{\tab}{Tab.}
\newcommand{\tabs}{Tabs.}
\newcommand{\fig}{Fig.}
\newcommand{\figs}{Figs.}
\newcommand{\Fig}{Fig.}
\newcommand{\Figs}{Figs.}
\newcommand{\alg}{Alg.}
\newcommand{\page}{page}


\newcommand{\dphidn}{\frac{\partial\phi}{\partial\mathbf{n}}}
%\newcommand{\pdiff}[2]{\frac{\partial{#1}}{\partial{#2}}}
\newcommand{\ab}[1]{{\ensuremath{\langle #1\rangle}}}

\newcommand{\Iota}{\ensuremath{\mathrm{I}}}
\newcommand{\gapfunc}{\ensuremath{\mathbf{g}}}
\newcommand{\lagrange}{\ensuremath{\mathcal{L}}}
\newcommand{\action}{\ensuremath{\mathcal{A}}}
\newcommand{\VE}{vertex-edge}


\newtheorem{thm}{Theorem}[section]
\newtheorem{cor}{Corollary}[section] % No numbers....  (?)
\newtheorem{dfn}{Definition}[section]
\newtheorem{example}{Example}[section]
\newtheorem{proposition}{Prop.}[section]

\newcommand{\nth}{\ensuremath{n^{\text{th}}}}
\newcommand{\mth}{\ensuremath{m^{\text{th}}}}
\newcommand{\ith}{\ensuremath{i^{\text{th}}}}
\newcommand{\jth}{\ensuremath{j^{\text{th}}}}

\newcommand{\ee}[1]{\ensuremath{\times 10^{#1}}}


\newcommand{\Rone}{\ensuremath{\mathbb{R}^1}}
\newcommand{\goesto}{\ensuremath{\to}}
\newcommand{\imples}{\ensuremath{\ \Rightarrow\ }}


%  A couple of macros...
\def\matlab{{\tt Matlab}}
\def\maple{{\tt Maple}}


%Give a little extra space in the equation arrays because
%of all the sums and fractions.
\setlength{\jot}{.2in}



%%%%%  FIXME: Move this up:
\def\smallfig{55mm}


\newcommand{\Comment}[1]{\textcolor{red}{#1}}


\pagestyle{plain}

\begin{document}


\title{TOS Message format issues}
%\author{D. M. Doolin\thanks{%
%Post-doctoral researcher, University of California,
%Civil and Environmental Engineering,
%2108 Shattuck Ave., Berkeley, CA 94720-1716}}
\date{\today}
\maketitle


\section*{1}

\begin{verbatim}
From get@eecs.berkeley.edu Wed Mar 24 13:39:39 2004
Date: Tue, 23 Mar 2004 12:01:45 -0800
From: Gilman Edwin Tolle <get@eecs.berkeley.edu>
To: tinyos-users@Millennium.Berkeley.EDU
Subject: [Tinyos-users] A conceptual question about Active Messages

I've been thinking a bit about a question, and I'd like to get the 
opinions of the list:

Are Active Message IDs intended to be in 1-1 correspondence with message 
formats?

Most of what I've seen indicates yes:
* We give names like AM_SURGEMSG to numeric IDs.
* The mig-generated Java message classes contain explicit AM IDs.
* TinyOS components expect to be able to correctly extract pieces from
   messages based only on the "type" field in the container TOS_Msg
   structure.

However, some messages have been designed to be "generic container" 
messages, of which I can think of two: BcastMsg and MultihopMsg. These 
messages seem to have no uniquely associated type, and to use them, the 
programmer must take explicit action to copy the type from the "inner 
messages" they contain to the outer TOS_Msg containing them. In a TinyOS 
component, this action requires an extra line of wiring directly to 
GenericComm. In a Java app, this requires passing the inner message 
containing the correct type to the registerListener() function, then 
typecasting each incoming message to the "outer message". In addition, 
to correctly send a container message, Java developers must hand-edit 
the generated container message class file to set the type because none 
is automatically included.

This departure from the 1-1 model makes several techniques much more 
challenging, in addition to requiring extra effort by the programmer to 
understand and use. For instance, creating an independent Java program 
that can receive all MultihopMsgs passed over the UART, parse the 
MultihopMsg part, and then keep statistics is not possible in a generic 
way. In addition, automatically and recursively extracting the parts 
from a nested message for more effective message logging is impossible 
without explicit type information at each level of nesting.

So, I wonder why this decision was made, and whether it would be a good 
idea to reverse it.

Reassociating container messages with a unique type is easy enough, and 
adding an internal type field to each container message can re-establish 
the missing type information about the internal message. Parameterized 
TinyOS interfaces can be dispatched just as easily on inner type fields 
as on outer type fields, and this would obviate the need to wire 
GenericComm directly. Java interface classes will no longer need 
hand-editing or misdirection, and much more flexible message parsing 
becomes possible.

If I've misunderstood anything, let me know.

Gil

\end{verbatim}



\section*{2}

\begin{verbatim}
From cssharp@eecs.berkeley.edu Wed Mar 24 13:39:52 2004
Date: Tue, 23 Mar 2004 12:56:01 -0800
From: Cory Sharp <cssharp@eecs.berkeley.edu>
To: tinyos-users@Millennium.Berkeley.EDU
Cc: Alec Woo <awoo@cs.berkeley.edu>, Gilman Edwin Tolle <get@eecs.berkeley.edu>
Subject: Re: [Tinyos-users] A conceptual question about Active Messages

I took me a second to realize what you're arguing for, but I think I'll 
agree.  You propose that for a TOS_Msg.data structured something like this

     /-- TOS_Msg.data ------------------------\
     AM1 [header1] AM2 [header2] ... AMn [body]

then all AM's (AM1, AM2, ..., AMn) should live is a globally unique namespace.

This directly addresses a problem we had developing PEG's routing where each 
AM generated a new namespace for the nested AM's within it.  This required 
wiring information be available to mig, etc (which I never did and just 
created the extremely grotesque peg.m in matlab).  Placing AM's in a global 
namespace removes that need for wiring, just looping over consequitive AM's 
in the message.

One problem, managing and posibly exhausing that namespace are issues for a 
large project.  In PEG, we had more than 50 Config values.  Each config 
value effectively had it's own AM within the Config namespace, and it was 
hard enough to coordinate AM's just for those Config values.

It might be worth considering solutions to namespace exhaustion, otherwise 
you might hit a hard wall at just the wrong time -- deep into development. 
Otherwise, I think it's a great idea.

- Cory
\end{verbatim}

\section*{3}

\begin{verbatim}
From get@eecs.berkeley.edu Wed Mar 24 13:40:02 2004
Date: Tue, 23 Mar 2004 13:33:24 -0800
From: Gilman Edwin Tolle <get@eecs.berkeley.edu>
To: Cory Sharp <cssharp@eecs.berkeley.edu>
Cc: tinyos-users@Millennium.Berkeley.EDU, Alec Woo <awoo@cs.berkeley.edu>
Subject: Re: [Tinyos-users] A conceptual question about Active Messages

Let me see if I understand what you mean.

The current system looks like this:

TOS_Msg.type = AM_SURGEMSG
TOS_Msg.data ---------> MultihopMsg.fields
                         MultihopMsg.data -----> SurgeMsg.fields

What we have now is a single global namespace for all nested 
combinations of messages. That is, a TOS_Msg with type AM_SURGEMSG must 
contain a MultihopMsg containing a SurgeMsg.


What I'm arguing for is something more like this:

TOS_Msg.type = AM_MULTIHOPMSG
TOS_Msg.data ---------> MultihopMsg.fields
                         MultihopMsg.type = MH_SURGEMSG
                         MultihopMsg.data ------> SurgeMsg.fields

I'd like to see explicit type information at each level, with a separate 
namespace for each level of containment. Following what Alec said, this 
allows a cleaner encapsulation of services, and requires less 
coordination between users of protocols at different layers.

Your message seems to imply something like my second example, but with 
each nested type having a globally unique value. I would actually argue 
that as long as the type of each contained message is always stored by 
the container message in a field named "type", possibly with a fixed 
offset from the start of the message, then we could have a separate 
namespace at each level and still perform automatic recursive 
processing. Each message type will still have a uniquely associated ID, 
but it needs only to be interpreted in the context of its containing 
messages.

I'm not sure why this would require wiring information in mig, and in 
fact, because each message is wholly independent of its container, it 
should require even less information about the containment hierarchy.

If a unique name is required for each distinct nested set of messages, 
as in the current model, then the namespace would probably be exhausted 
sooner than if a name was only associated with each message. If we 
allowed overlap, such that two messages could share the same ID as long 
as one was intended to be sent out within a MultihopMsg and the other to 
be sent out within a BcastMsg, then we would have even less chance of 
namespace exhaustion.

Gil

\end{verbatim}

\section*{4}

\begin{verbatim}
From pal@eecs.berkeley.edu Wed Mar 24 13:40:14 2004
Date: Tue, 23 Mar 2004 15:03:22 -0800
From: Philip Levis <pal@eecs.berkeley.edu>
To: Gilman Edwin Tolle <get@eecs.berkeley.edu>
Cc: tinyos-users@Millennium.Berkeley.EDU, Alec Woo <awoo@cs.berkeley.edu>,
     Cory Sharp <cssharp@eecs.berkeley.edu>
Subject: Re: [Tinyos-users] A conceptual question about Active Messages

On Tuesday, March 23, 2004, at 01:33 PM, Gilman Edwin Tolle wrote:

> Let me see if I understand what you mean.
>
> The current system looks like this:
>
> TOS_Msg.type = AM_SURGEMSG
> TOS_Msg.data ---------> MultihopMsg.fields
>                         MultihopMsg.data -----> SurgeMsg.fields
>
> What we have now is a single global namespace for all nested 
> combinations of messages. That is, a TOS_Msg with type AM_SURGEMSG 
> must contain a MultihopMsg containing a SurgeMsg.
>
>
> What I'm arguing for is something more like this:
>
> TOS_Msg.type = AM_MULTIHOPMSG
> TOS_Msg.data ---------> MultihopMsg.fields
>                         MultihopMsg.type = MH_SURGEMSG
>                         MultihopMsg.data ------> SurgeMsg.fields
>
> I'd like to see explicit type information at each level, with a 
> separate namespace for each level of containment. Following what Alec 
> said, this allows a cleaner encapsulation of services, and requires 
> less coordination between users of protocols at different layers.
>

The issue is the possibility of wasted space. If you have 256 AM IDs 
available, why add another level of dispatch? For one level, this makes 
sense; but when you have fine-grained layering (as Cory's stuff does), 
this becomes onerous. E.g.:

AM -> ReliableAM (retransmissions) -> Routing -> SurgeMsg -> 
SurgeMsgType

I recall that we wanted to do this way back when with mig, to allow 
nested packet types, but there was this tricky corner case where it 
would break down.


> Your message seems to imply something like my second example, but with 
> each nested type having a globally unique value. I would actually 
> argue that as long as the type of each contained message is always 
> stored by the container message in a field named "type", possibly with 
> a fixed offset from the start of the message, then we could have a 
> separate namespace at each level and still perform automatic recursive 
> processing. Each message type will still have a uniquely associated 
> ID, but it needs only to be interpreted in the context of its 
> containing messages.
>
> I'm not sure why this would require wiring information in mig, and in 
> fact, because each message is wholly independent of its container, it 
> should require even less information about the containment hierarchy.
>
> If a unique name is required for each distinct nested set of messages, 
> as in the current model, then the namespace would probably be 
> exhausted sooner than if a name was only associated with each message. 
> If we allowed overlap, such that two messages could share the same ID 
> as long as one was intended to be sent out within a MultihopMsg and 
> the other to be sent out within a BcastMsg, then we would have even 
> less chance of namespace exhaustion.

Namespace exhaustion isn't nearly as great a problem as namespace 
management and efficient structuring. The current getBuffer() approach 
pushes all of this to run-time, when really it could be done at compile 
time with some dynamic component generation (as Cory has done). This 
also gives you better type checking.

However, I think that you're right, at a coarse grain. One could 
imagine, for certain major services, such as routing and dissemination, 
you may add another single level of dispatch. The problem is, when done 
at a fine grain, things go ill.

Phil

\end{verbatim}

\section*{5}

\begin{verbatim}
From mdw@eecs.harvard.edu Wed Mar 24 13:40:28 2004
Date: 23 Mar 2004 18:03:25 -0500
From: Matt Welsh <mdw@eecs.harvard.edu>
To: Philip Levis <pal@eecs.berkeley.edu>
Cc: tinyos-users@Millennium.Berkeley.EDU,
     Gilman Edwin Tolle <get@eecs.berkeley.edu>
Subject: Re: [Tinyos-users] A conceptual question about Active Messages

I like the approach that Cory has taken with his protocol code, and
think this could be extended further. Most RPC systems support a notion
of a stub generator that emits specialized code to marshal/unmarshal
message types on the wire. Generating appropriate NesC components to
handle this as well as the buffer management would be fantastic. It
could even be extended to support specific retransmission policies, etc.
moving the complexity from runtime to compile time. Good course project
idea...

\end{verbatim}

\section*{6}

\begin{verbatim}
From pal@eecs.berkeley.edu Wed Mar 24 13:40:37 2004
Date: Tue, 23 Mar 2004 15:26:02 -0800
From: Philip Levis <pal@eecs.berkeley.edu>
To: Matt Welsh <mdw@eecs.harvard.edu>
Cc: tinyos-users@Millennium.Berkeley.EDU,
     Gilman Edwin Tolle <get@eecs.berkeley.edu>
Subject: Re: [Tinyos-users] A conceptual question about Active Messages

On Tuesday, March 23, 2004, at 03:03 PM, Matt Welsh wrote:

> I like the approach that Cory has taken with his protocol code, and
> think this could be extended further. Most RPC systems support a notion
> of a stub generator that emits specialized code to marshal/unmarshal
> message types on the wire. Generating appropriate NesC components to
> handle this as well as the buffer management would be fantastic. It
> could even be extended to support specific retransmission policies, 
> etc.
> moving the complexity from runtime to compile time. Good course project
> idea...

That's what MIG does, except that the packing/unpacking is merely from 
Java object representations to TinyOS structures. RPC is quite a 
different beast; the purpose of marshalling/unmarshalling there has 
much to do with pointers and platform independence; the appearance of a 
function call is very different than packing a message structure.

The whole issue of platform dependence on issues such as byte packing 
(between motes and TOSSIM, for example, were it to run on a Motorola 
instead of an Intel) has raised the question of whether nesC should 
have a message type. Field accesses would be transformed into the right 
packing/unpacking operations. There would be some limitations over 
standard structs (e.g., no taking pointers). The idea is to do it at 
the language level, instead of through components. Once you do this, 
all of the buffer management issues go away.

Phil

\end{verbatim}

\section*{7}

\begin{verbatim}
From get@eecs.berkeley.edu Wed Mar 24 13:40:47 2004
Date: Tue, 23 Mar 2004 15:44:23 -0800
From: Gilman Edwin Tolle <get@eecs.berkeley.edu>
To: tinyos-users@Millennium.Berkeley.EDU, Alec Woo <awoo@eecs.berkeley.edu>,
     Cory Sharp <cssharp@eecs.berkeley.edu>,
     Philip Levis <pal@eecs.berkeley.edu>
Subject: Re: [Tinyos-users] A conceptual question about Active Messages


Your wasted space issue makes a lot of sense to me. For deeply-nested 
messages, the extra overhead of storing a message type at each level may 
be unjustifiable, especially if a given message is not likely to be used 
to contain more than one different submessage.

Perhaps there could be a compromise, with some sort of prefix code 
system. For instance, given your example below, I would consider it 
highly likely that many different components would like to use a Routed 
Reliable Active Messages, so we could associate that three-message 
prefix with a single AM type. Automatic recursive unmarshalling would 
still be possible with such a prefix scheme, but might require a bit 
more "smarts" in the decoder.

Conversely, I don't consider it very likely that different applications 
would put their own messages under SurgeMsg, so perhaps SurgeMsg is not 
really a container msg and thus does not even need its own type at all.

As a sidenote, getBuffer() doesn't seem to do anything that couldn't be 
done at compile time. All it does is make a call to offsetof() to find 
the start of the data field, and I'd imagine that's a compile-time macro 
anyway.

Gil

\end{verbatim}

\section*{8}

\begin{verbatim}
From pal@eecs.berkeley.edu Wed Mar 24 13:40:59 2004
Date: Tue, 23 Mar 2004 16:00:25 -0800
From: Philip Levis <pal@eecs.berkeley.edu>
To: Gilman Edwin Tolle <get@eecs.berkeley.edu>
Cc: tinyos-users@Millennium.Berkeley.EDU,
     Cory Sharp <cssharp@eecs.berkeley.edu>, Alec Woo <awoo@eecs.berkeley.edu>
Subject: Re: [Tinyos-users] A conceptual question about Active Messages

On Tuesday, March 23, 2004, at 03:44 PM, Gilman Edwin Tolle wrote:

> Conversely, I don't consider it very likely that different 
> applications would put their own messages under SurgeMsg, so perhaps 
> SurgeMsg is not really a container msg and thus does not even need its 
> own type at all.
>

I sort of meant that there are different kind of Surge messages; sensor 
data messages, route data messages, etc.

> As a sidenote, getBuffer() doesn't seem to do anything that couldn't 
> be done at compile time. All it does is make a call to offsetof() to 
> find the start of the data field, and I'd imagine that's a 
> compile-time macro anyway.

Absolutely. But it requires a bit of language/compiler support to fold 
the constants. The idea was basically, "How can we get the behavior we 
want, without doing a lot of work on what might be the wrong thing, or 
introducing compilation complexities?" That is, just use C. It's not 
the right long-term solution.

Phil

\end{verbatim}

\section*{9}

\begin{verbatim}
From mdw@eecs.harvard.edu Wed Mar 24 13:41:09 2004
Date: 23 Mar 2004 20:58:04 -0500
From: Matt Welsh <mdw@eecs.harvard.edu>
To: Philip Levis <pal@eecs.berkeley.edu>
Cc: tinyos-users@Millennium.Berkeley.EDU,
     Gilman Edwin Tolle <get@eecs.berkeley.edu>
Subject: Re: [Tinyos-users] A conceptual question about Active Messages

> The whole issue of platform dependence on issues such as byte packing 
> (between motes and TOSSIM, for example, were it to run on a Motorola 
> instead of an Intel) has raised the question of whether nesC should 
> have a message type. Field accesses would be transformed into the right 
> packing/unpacking operations. There would be some limitations over 
> standard structs (e.g., no taking pointers). The idea is to do it at 
> the language level, instead of through components. Once you do this, 
> all of the buffer management issues go away.

Right - that is *exactly* what I am advocating.

M.

\end{verbatim}

\section*{10}

\begin{verbatim}
From pal@eecs.berkeley.edu Wed Mar 24 13:41:51 2004
Date: Wed, 24 Mar 2004 11:31:13 -0800
From: Philip Levis <pal@eecs.berkeley.edu>
To: Matt Welsh <mdw@eecs.harvard.edu>
Cc: tinyos-users@Millennium.Berkeley.EDU,
     Gilman Edwin Tolle <get@eecs.berkeley.edu>
Subject: Re: [Tinyos-users] A conceptual question about Active Messages

On Tuesday, March 23, 2004, at 05:58 PM, Matt Welsh wrote:

>> The whole issue of platform dependence on issues such as byte packing
>> (between motes and TOSSIM, for example, were it to run on a Motorola
>> instead of an Intel) has raised the question of whether nesC should
>> have a message type. Field accesses would be transformed into the 
>> right
>> packing/unpacking operations. There would be some limitations over
>> standard structs (e.g., no taking pointers). The idea is to do it at
>> the language level, instead of through components. Once you do this,
>> all of the buffer management issues go away.
>
> Right - that is *exactly* what I am advocating.

Ah, excellent. I misunderstood. Sorry.

Phil

\end{verbatim}


\end{document}


