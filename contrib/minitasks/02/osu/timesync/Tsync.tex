\documentclass[11pt]{article}
\usepackage{palatino,helvetic,courier,mathptm} % favor PDF translation
%  make box around each page
\special{!userdict begin /bop-hook{gsave 0.12 0.56 1.00 setrgbcolor
 72 72 moveto 0 640 rlineto 465 0 rlineto 0 -640 rlineto closepath 
 4 setlinewidth .96 setgray fill stroke grestore} def end}
\def\Tsync{\textsf{Tsync}}
\begin{document}
\title{NestArch: Prototype Time Synchronization Service}
\author{Ted Herman, University of Iowa, \texttt{herman@cs.uiowa.edu}}
\date{6 January 2003}
\maketitle
\begin{abstract}
This document is a brief explanation of a rudimentary time
service (\Tsync) developed for the NEST Challenge Architecture.  The 
document describes the basic design, performance expectations for
\Tsync, and its TinyOS implementation.
\end{abstract}

\section{Introduction}

Time synchronization in a distributed, real-time, system is a 
basic service.  It can be useful for the NEST Challenge Project in
several ways: calculation of velocities, coordination of 
middleware services, and resource allocation algorithms.  Initial
specifications for a time synchronization service were proposed in
the Fall of 2002;  subsequent experience implementing time 
synchronization in the mote architecture, along with the pressure
of project deadlines, have motivated a few modifications to the
initial specification.  However the spirit of the original
specification remains in place:  global system time is represented
by an integer, denominated in 30.5176 $\mu$sec units 
(\emph{ie}, 32,768 ticks per second).  The remainder of this
document explains the basics of the initial implementation, 
consequences of its simple design with respect to expected 
performance, and avenues for improvement.  Throughout this document
we refer to the time synchronization service as \Tsync.
\section{The Interface}
Users of \Tsync\ will be happy to see that it offers a simple
interface.  In addition to the standard control interface 
\texttt{StdControl} (used by many components), \Tsync\ supports
the \texttt{Time} interface.  This interface provides two 
commands for obtaining the current time, \texttt{getGlobalTime} and
\texttt{getLocalTime}.  The caller of these commands supplies a
pointer to a \texttt{timeSync} structure, which is nothing more
than a \texttt{uint32}-integer to store the time (in future, this
structure could become richer).  In most practice, there is no difference
between these two commands;  during initialization of a mote network,
\texttt{getGlobalTime} could return a \texttt{FAIL} status in case
mote clocks have not yet been synchronized, whereas \texttt{getLocalTime}
always succeeds.
\par
In the initial draft of the time synchronization time was specified
as a \texttt{uint48}-integer;  elsewhere, Su Ping proposed that 
time be represented by \texttt{uint64}-integers.  We chose \texttt{uint32}
for the present because it is adequate for the challenge demonstration
and can easily be changed later.  Note that \texttt{uint32}-valued time
corresponds to about 36 hours. 
\par
Earlier in our efforts to specify a time synchronization service, we
considered the possibility of regional time bases (local time, somewhat
like a ``time zone'') such as the Reference Broadcast System 
\cite{elson} uses.  Other possibilities not contemplated include
using time intervals (a pair of numbers denotes an interval, and
this interval contains the true time), vectors or matrices, and even
graph-theoretic structures.  Also, it is reasonable to consider 
enhancing the \texttt{timeSync} structure someday to have indications
of the quality of the clock returned by a call to the service.  
In addition, the interface could have interactions for long-term
energy saving, tuning, and other factors.  

\section{Some Facts about Mote Clocks}

We explain some of the basics about how motes deal with clocks,
since this influences our implementation and explains some of the
performance consequences.  
\par
Motes do not have ``clocks'' as such;  rather, some registers 
of the processor can be programmed (subject to several limitations)
to increment with each processor cycle (\emph{eg} at 4 MHz) and 
generate an interrupt upon reaching a maximum count.  Therefore we've
got to simulate a clock.  Since our desired clock precision 
is 32768 Hz, our simulated clock should ideally advance once 
per 122.07 processor cycles.  This turns out to be impractical
for two reasons.  First, the effort of processing interrupts to
increment a software counter is too expensive (competing with 
other processing needs);  second, the processor is not so flexible
in how it can program its hardware counters.  
\par
The hardware counter behind the simulated clock can be programming
in eight different granularities, and for each of these granularities
there is a multiplier.  If the granularity is set at 32768 Hz,
then an interrupt will occur after 30.5176 $\mu$sec (minimum) up to
7.78198 msec (maximum), depending on the value of the 8-bit multiplier.
Other granularities are 4096, 1024, 512, 256, 128, 32, and 0 Hz.
This implies that if we desire an interrupt to occur about every
second, either a multiplier of 32 with a granularity of 32 can be 
used or a multiplier of 128 with a granularity of 128 can be used;
higher granularities won't work if we want one clock interrupt per
second.  Why is this important?  It has consequences for our 
desired implementation of a time service with 32 KHz precision. 
It is just too expensive to actually increment the clock at 32768 Hz;
however if the hardware counter is programmed to interrupt once per
second, then each interrupt will trigger 
\begin{quote}
\verb'clock = clock + 32768'
\end{quote}
the overhead is acceptably low, and we have the desired result!?  
Well, not really.  What if some application queries the time service
for the current clock \emph{between} the counter-driven interrupts?
The accuracy will be around half a second on average -- we might as
well use a clock whose units are half seconds.  Fortunately, the 
story doesn't end here.
\par
The processor can also \emph{read} the value of the hardware counter,
before it has generated an interrupt.  Thus an instantaneous reading
of time is possible.  We implemented this approach for \Tsync.  Its
accuracy is, however, dependent on the granularity of the hardware
counter.  If the timer has been set to fire once per second, then
it sets the granularity to 128 Hz, so an instantaneous reading will
have an accuracy of about 3.9 milliseconds on average (and each 
counter-driven interrupt adds 256 to the counter).  We anticipate this
to be the norm for timer settings (firing once per second).  When
the timer is set to higher frequency, say 10 millisecond firing, then
the granularity does become 32768 Hz, but this is expensive and only
to be used sparingly.  Applications that need to measure the difference
between two times obtained from \Tsync\ therefore should be designed
in the context of timer (\emph{ie} the TinyOS \texttt{Timer} component)
settings.
\par
In developing the prototype \Tsync\ implementation, we calibrated
its software clock using a GPS-delivered pulse-per-second signal.
We observed that \Tsync\ time is about 1.5\% slower than real time
at typical timer settings;  at the highest granularity, the clock
ran about 3\% slower than real time.  We speculate that most of
this drift is due to delays of interrupt processing.  In the initial 
version, there is no correction to the software clock to compensate
for this observed phenomenon.

\section{Basic Design}

The constraints for our basic design are:  (1) simplicity;  
(2) some fault tolerance;  (3) no built-in dependence on base stations
or specialized mote roles.  
\par
Our design implicitly elects the mote with the smallest identifier
to be the ``root clock''.  The root clock periodically announces the
value of its \Tsync-time; other motes copy the root clock.  Thus
this is a \emph{push} design (whereas NTP uses a pull method to 
synchronize).  Only one message type (\texttt{Beacon}) is used
by the \texttt{Tsync} component.
\par
Each non-root mode chooses, among its set of neighbors, one with
least distance (measured in hops) to the root.  The neighbor chosen
supplies its clock as an approximation to the root time.  The set of
neighbors is determined by building a table based on received 
beacon messages.  Beacon messages contain fields to name the root
identifier, the number of hops to the root, the sender's clock at
the instant of sending, and a few other values.  In the initial
implementation, there is no correction for the latency of sending
and receiving a beacon message (we estimate the latency to be
about 36 milliseconds).
\par
What happens if a mote dies, or a link is somehow lost between
motes?  \Tsync\ uses an aging technique to maintain the neighbor
table.  If a beacon hasn't been received for too long, then a 
neighbor is removed from the table.  If, as a result, no neighbor
can offer a path to the current root, a mote will adopt a new
root.  Technically this is done by the well-known ``count to 
infinity'' method of distance-vector routing protocols.  The 
\Tsync\ implementation uses a hard-coded limit on feasible hop
distances to force the eventual extinguishing of path information.
Thus if a root mote dies, eventually a new root emerges.  
\par
What happens if a new mote, which has a lower identifier than
any previously known, enters the system?  We didn't want the 
entire system to adopt the new mote's clock as it became root.
Therefore \Tsync's implementation refuses to accept the clock
in a beacon message unless that beacon's sender has a nonempty
neighborhood;  and a mote with an empty neighborhood accepts the
clock of the first beacon (with nonempty neighborhood) that it
receives.
\par
To be sure, there are many other failure cases that \Tsync\ 
does not cover;  better fault tolerance is just one of the
areas of improvement for future versions of \Tsync.

\section{Performance}

As the reader can see from the basic design, \Tsync\ doesn't
offer anything in the way of performance guarantees.  We 
guess that clocks will be within about $36\cdot h$ milliseconds
of the root mote's, where $h$ is the number of hops to the
root.  The clocks are adjusted whenever a new root is detected
or whenever some neighbor has a smaller clock than the current
time.  Thus \Tsync's clocks are \emph{not monotonic} in the
initial version.

\begin{thebibliography}{0}

\bibitem{elson} J Elson, L Girod, D Estrin.  Fine-grained
network time synchronization using reference broadcasts.  May 2002.

\end{thebibliography}

\end{document}
